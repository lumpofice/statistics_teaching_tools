\section{Data Types}

Data can fall into $4$ different types:
\begin{itemize}
    \item Nominal
    \item Ordinal
    \item Interval
    \item Ratio
\end{itemize}
across two different categories:
\begin{itemize}
    \item Qualitative
    \item Quantitative
\end{itemize}
Qualitative data indicate the presence of a characteristic. The specimen in question either does or does not have the characteristic. 
\begin{example}
The specimen in question, say a person, either does or does not have brown eyes.
\end{example}

\begin{dataset}
The player's name, `Player', is Nominal data. The set of numbers out to the left of the `Player' column represents an index. An index is used to indicate an order for a given element, which makes indices Ordinal data. \\[1ex]
\input{data_types/players_last_name_d_nominal}\\[4ex]
\end{dataset}

\begin{dataset}
The player's name, `Player', is Nominal data. The year at which the player began their career, `From', is Interval data. The weight of the player, `Wt', is Ratio data.\\[1ex]
\input{data_types/players_last_name_d_nominal_interval_ratio}\\[4ex]
\end{dataset}

\begin{dataset}
The points ranking of the player, `Rk', is ordinal data. The player's name, `Player', is Nominal data type. The total points a player scored during the 2021-2022 season, `PTS', is Ratio data.\\[1ex]
\begin{tabular}{llr}
\toprule
{} &                 Player &  PTS▼ \\
Rk &                        &       \\
\midrule
1  &             Trae Young &  2155 \\
2  &          DeMar DeRozan &  2118 \\
3  &            Joel Embiid &  2079 \\
4  &           Jayson Tatum &  2046 \\
5  &           Nikola Jokić &  2004 \\
6  &  Giannis Antetokounmpo &  2002 \\
7  &            Luka Dončić &  1847 \\
8  &           Devin Booker &  1822 \\
9  &     Karl-Anthony Towns &  1818 \\
10 &       Donovan Mitchell &  1733 \\
11 &           LeBron James &  1695 \\
12 &           Kevin Durant &  1643 \\
13 &            Zach LaVine &  1635 \\
14 &          Stephen Curry &  1630 \\
15 &          Miles Bridges &  1613 \\
16 &              Ja Morant &  1564 \\
17 &           Jaylen Brown &  1559 \\
18 &          Pascal Siakam &  1551 \\
19 &        Anthony Edwards &  1533 \\
20 &            LaMelo Ball &  1508 \\
\bottomrule
\end{tabular}
\\[4ex]
\end{dataset}