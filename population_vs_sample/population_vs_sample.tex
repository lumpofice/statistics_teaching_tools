\section{Population vs. Sample}

\begin{definition}
A population is the set of all possible data that meet a specific definition. 
\end{definition}

\begin{definition}
A sample is a subset of a given population.
\end{definition}

\begin{example}
Let the definition for the population be
\begin{center}
    The set of ages for all teachers who work at school A.
\end{center}
Our population is 
\begin{align*}
    G = \{x_{i}: x_{i} \hspace{4pt} \text{is the age for teacher} \hspace{4pt} i \hspace{4pt} \text{who works at school A}\}
\end{align*}
One sample of $G$ could be the set of ages for all science teachers who work at school A
\begin{align*}
    S = \{x_{i}: x_{i} \hspace{4pt} \text{is the age for science teacher} \hspace{4pt} i \hspace{4pt} \text{who works at school A}\}
\end{align*}
Another sample of $G$ could be the set of ages for all chemistry teachers who work at school A
\begin{align*}
    C = \{x_{i}: x_{i} \hspace{4pt} \text{is the age for chemistry teacher} \hspace{4pt} i \hspace{4pt} \text{who works at school A}\}
\end{align*}
\end{example}

